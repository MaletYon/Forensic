\chapter{Outils et Techniques Avancées}
\epigraph{"By approaching each case methodically, you can evaluate the evidence thoroughly and document the chain of evidence, or chain of custody..."}{- Amelia Phillips}
\section{Arsenal de l'Investigateur Moderne}
\subsection{Acquisition et Imagerie}
\begin{lstlisting}[language=Python, caption=Script d'acquisition avec validation]
#!/usr/bin/env python3
import hashlib
import subprocess
import time

def forensic_acquisition(source, destination):
    """
    Acquisition forensique avec validation d'intégrité
    """
    # Phase 1: Pre-acquisition hash
    print("[*] Computing source hash...")
    source_hash = compute_hash(source)
    
    # Phase 2: Acquisition with dd
    print("[*] Starting acquisition...")
    start_time = time.time()
    cmd = f"dd if={source} of={destination} bs=65536 conv=noerror,sync"
    subprocess.run(cmd, shell=True)
    
    # Phase 3: Post-acquisition validation
    print("[*] Validating image...")
    dest_hash = compute_hash(destination)
    
    # Phase 4: Report
    elapsed = time.time() - start_time
    print(f"[+] Acquisition complete in {elapsed:.2f} seconds")
    print(f"[+] Source SHA-256: {source_hash}")
    print(f"[+] Dest SHA-256: {dest_hash}")
    print(f"[+] Integrity: {'VERIFIED' if source_hash == dest_hash else 'FAILED'}")

def compute_hash(file_path):
    sha256 = hashlib.sha256()
    with open(file_path, 'rb') as f:
        for chunk in iter(lambda: f.read(65536), b""):
            sha256.update(chunk)
    return sha256.hexdigest()
\end{lstlisting}

\subsection{Analyse de Mémoire Avancée}
\begin{lstlisting}[language=Python, caption=Volatility 3 Plugin Custom]
import volatility3.plugins.windows as windows
from volatility3.framework import interfaces, renderers

class SuspiciousProcessDetector(interfaces.plugins.PluginInterface):
    """Détecte les processus suspects basés sur des heuristiques"""
    
    def run(self):
        # Analyse heuristique
        suspicious_indicators = [
            "cmd.exe spawned by winword.exe",
            "powershell.exe with encoded command",
            "rundll32.exe without arguments",
            "svchost.exe from wrong path"
        ]
        
        for proc in self.list_processes():
            if self.is_suspicious(proc, suspicious_indicators):
                yield (proc.pid, proc.name, proc.ppid, "SUSPICIOUS")
\end{lstlisting}

\section{Techniques d'Anti-Anti-Forensique}
\subsection{Contournement de Chiffrement}
\textbf{Techniques légales uniquement avec autorisation judiciaire}

\begin{enumerate}
\item \textbf{Cold Boot Attack}
\begin{itemize}
\item Récupération de clés en mémoire
\item Refroidissement RAM à -50°C
\item Extraction dans les 10 minutes
\end{itemize}

\item \textbf{Evil Maid Attack}
\begin{itemize}
\item Installation de keylogger hardware
\item Modification du bootloader
\item Capture de passphrase
\end{itemize}

\item \textbf{DMA Attack}
\begin{itemize}
\item Utilisation Thunderbolt/FireWire
\item Accès direct mémoire
\item Bypass de l'OS
\end{itemize}
\end{enumerate}

\subsection{Détection de Techniques d'Obfuscation}
\begin{lstlisting}[language=Python, caption=Détection de stéganographie]
import numpy as np
from PIL import Image

def detect_lsb_steganography(image_path):
    """
    Détecte la stéganographie LSB par analyse statistique
    """
    img = Image.open(image_path)
    pixels = np.array(img)
    
    # Chi-square test on LSBs
    lsb_plane = pixels & 1
    expected_freq = len(lsb_plane.flatten()) / 2
    ones = np.sum(lsb_plane)
    zeros = len(lsb_plane.flatten()) - ones
    
    chi_square = ((ones - expected_freq)**2 +
                 (zeros - expected_freq)**2) / expected_freq
    
    # Threshold for suspicion
    if chi_square > 3.841:  # 95% confidence
        return "STEGANOGRAPHY DETECTED"
    return "CLEAN"
\end{lstlisting}

\section{Intelligence Artificielle en Investigation}
\subsection{Machine Learning pour Classification de Malware}
\begin{lstlisting}[language=Python, caption=Classificateur de malware]
from sklearn.ensemble import RandomForestClassifier
import pefile
import numpy as np

class MalwareClassifier:
    """
    Classificateur de malware basé sur les caractéristiques PE
    """
    
    def __init__(self):
        self.model = RandomForestClassifier(n_estimators=100)
        self.features = []
    
    def extract_features(self, pe_file):
        """Extraction de features depuis un PE"""
        pe = pefile.PE(pe_file)
        features = [
            pe.FILE_HEADER.NumberOfSections,
            pe.OPTIONAL_HEADER.SizeOfCode,
            pe.OPTIONAL_HEADER.AddressOfEntryPoint,
            len(pe.DIRECTORY_ENTRY_IMPORT) if hasattr(pe, 'DIRECTORY_ENTRY_IMPORT') else 0,
            self.calculate_entropy(pe_file)
        ]
        return np.array(features)
    
    def calculate_entropy(self, file_path):
        """Calcul d'entropie de Shannon"""
        with open(file_path, 'rb') as f:
            data = f.read()
        entropy = 0
        for i in range(256):
            freq = data.count(bytes([i])) / len(data)
            if freq > 0:
                entropy -= freq * np.log2(freq)
        return entropy
\end{lstlisting}

\subsection{Deep Learning pour Analyse Comportementale}
\begin{lstlisting}[language=Python, caption=Modèle LSTM pour analyse comportementale]
import tensorflow as tf
from tensorflow.keras import layers, models

def build_behavior_analysis_model():
    """
    Modèle LSTM pour analyse comportementale de processus
    """
    model = models.Sequential([
        layers.LSTM(128, return_sequences=True, 
                   input_shape=(None, 50)),  # 50 features
        layers.LSTM(64, return_sequences=True),
        layers.LSTM(32),
        layers.Dense(16, activation='relu'),
        layers.Dropout(0.2),
        layers.Dense(1, activation='sigmoid')  # Malicious/Benign
    ])
    
    model.compile(
        optimizer='adam',
        loss='binary_crossentropy',
        metrics=['accuracy']
    )
    
    return model
\end{lstlisting}