\chapter{Histoire de l'Investigation Numérique}
\epigraph{"Qui ne connaît pas l'histoire est condamné à la revivre. Dans notre domaine, cette répétition serait catastrophique."}{- Adaptation de George Santayana}
\section{Les Prémices (1970-1990)}
L'investigation numérique trouve ses racines dans les années 1970 avec l'apparition des premiers crimes informatiques. Le premier cas documenté remonte à \textbf{1971} avec l'affaire \textbf{"The Creeper"}, le premier ver informatique créé par Bob Thomas chez BBN Technologies. Cette attaque, bien qu'expérimentale, a posé les fondements de ce qui deviendrait la forensique numérique.

\subsection{L'Affaire du "414s" (1983)}
En 1983, un groupe de six adolescents de Milwaukee, surnommés les "414s" (d'après leur indicatif régional), ont pénétré dans 60 systèmes informatiques incluant le Los Alamos National Laboratory. Cette affaire a marqué un tournant:

\begin{itemize}
\item \textbf{Impact juridique}: Création du Computer Fraud and Abuse Act (1986) aux États-Unis
\item \textbf{Innovation technique}: Développement des premiers outils de traçage d'intrusion
\item \textbf{Leçon apprise}: Nécessité de préserver les preuves numériques de manière systématique
\end{itemize}

\section{L'Ère de la Professionnalisation (1990-2000)}
\subsection{L'Opération Sundevil (1990)}
Le 8 mai 1990, le Secret Service américain lance l'\textbf{Opération Sundevil}, la plus grande opération contre la cybercriminalité de l'époque:

\begin{itemize}
\item \textbf{Envergure}: 42 systèmes informatiques saisis dans 14 villes
\item \textbf{Innovation}: Première utilisation massive de techniques de préservation de preuves
\item \textbf{Problème révélé}: Manque de standardisation dans la collecte de preuves
\end{itemize}

Cette opération a révélé le besoin crucial de méthodologies standardisées, conduisant à la création de l'\textbf{International Organization on Computer Evidence (IOCE)} en 1995.

\subsection{Le Cas Kevin Mitnick (1995)}
L'arrestation de Kevin Mitnick le 15 février 1995 représente un jalon majeur:

\begin{itemize}
\item \textbf{Techniques utilisées}: Analyse de métadonnées, corrélation temporelle, traçage IP
\item \textbf{Expert clé}: Tsutomu Shimomura, qui a développé des techniques de honeypot
\item \textbf{Héritage}: Établissement du principe de "chain of custody" numérique
\end{itemize}

\section{L'Ère de la Standardisation (2000-2010)}
\subsection{L'Affaire Enron (2001)}
La faillite d'Enron a révolutionné l'e-discovery:

\begin{itemize}
\item \textbf{Volume}: 500 000 documents électroniques analysés
\item \textbf{Innovation}: Développement d'outils d'analyse automatisée (précurseurs du TAR - Technology Assisted Review)
\item \textbf{Impact}: Amendements aux Federal Rules of Civil Procedure (2006) pour l'e-discovery
\end{itemize}

\subsection{L'Affaire Gary McKinnon (2002)}
Le hacker britannique accusé d'avoir infiltré 97 serveurs militaires américains:

\begin{itemize}
\item \textbf{Durée de l'investigation}: 7 ans
\item \textbf{Technique clé}: Analyse des journaux distribués sur plusieurs fuseaux horaires
\item \textbf{Innovation}: Développement de techniques de corrélation multi-juridictionnelle
\end{itemize}

\section{L'Ère du Big Data et du Cloud (2010-2020)}
\subsection{L'Affaire Silk Road (2013)}
L'arrestation de Ross Ulbricht et la fermeture de Silk Road:

\begin{itemize}
\item \textbf{Innovation technique}: Analyse blockchain forensique
\item \textbf{Volume}: 144,000 bitcoins saisis
\item \textbf{Méthode clé}: Corrélation d'identités pseudonymes avec des métadonnées
\end{itemize}

\subsection{L'Affaire Panama Papers (2016)}
La plus grande fuite de données de l'histoire:

\begin{itemize}
\item \textbf{Volume}: 2.6 TB de données, 11.5 millions de documents
\item \textbf{Technique}: Graph analysis pour identifier les relations
\item \textbf{Impact}: Développement d'outils d'analyse de données massives
\end{itemize}

\section{L'Ère Post-Quantique et IA (2020-Présent)}
\subsection{L'Attaque SolarWinds (2020)}
Une des cyberattaques les plus sophistiquées:

\begin{itemize}
\item \textbf{Durée de compromission}: 9 mois non détectée
\item \textbf{Innovation}: Analyse comportementale basée sur l'IA
\item \textbf{Défi}: Attribution dans un contexte de techniques d'obfuscation avancées
\end{itemize}