\chapter{Droit Camerounais et Africain}
\epigraph{"Africa must develop its own digital legal framework that respects both international standards and local cultural realities."}{- Nnenna Ifeanyi-Ajufo}
\section{Cadre Législatif National}
\subsection{Loi N°2010/012 du 21 décembre 2010}
\textbf{Relative à la cybersécurité et la cybercriminalité}

\textbf{Dispositions clés}:

\begin{itemize}
\item \textbf{Article 3}: Définitions (système informatique, données)
\item \textbf{Articles 45-50}: Procédure de perquisition informatique
\item \textbf{Articles 60-65}: Conservation des données
\item \textbf{Articles 74-81}: Infractions et sanctions
\end{itemize}

\textbf{Autorités compétentes}:

\begin{itemize}
\item ANTIC (Agence Nationale des TIC)
\item Brigade de cybercriminalité
\item Parquet spécialisé
\end{itemize}

\subsection{Loi N°2010/013 du 21 décembre 2010}
\textbf{Régissant les communications électroniques}
\begin{itemize}
\item Obligations des opérateurs
\item Interception légale
\item Conservation des métadonnées (10 ans)
\end{itemize}


\subsection{Loi N°2024/017 du 23 décembre 2024}
\textbf{Régissant la protection des données à caractère personnel au Cameroun}

\begin{itemize}
\item Collecte
\item Traitement
\item Conservation 
\end{itemize}

\section{Procédure d'Investigation au Cameroun}
\subsection{Cadre Procédural}
\begin{verbatim}
Procédure type d'investigation numérique:

1. Plainte/Signalement
   ↓
2. Enquête préliminaire (OPJ)
   ↓
3. Ouverture information judiciaire
   ↓
4. Commission rogatoire pour expertise
   ↓
5. Expertise technique (expert agréé)
   ↓
6. Rapport d'expertise
   ↓
7. Audience (présentation des preuves)
   ↓
8. Jugement
\end{verbatim}

\subsection{Experts Agréés}
\textbf{Conditions} (Décret N°69/DF/544):

\begin{itemize}
\item Diplôme BAC+5 en informatique.
\item 5 ans d'expérience minimum.
\item Formation spécifique en forensique.
\item Agrément du Ministère de la Justice.
\end{itemize}

\section{Jurisprudence Camerounaise}
\subsection{Affaires Marquantes}
\textbf{Affaire CAMTEL c. X (2018)}:

\begin{itemize}
\item Première condamnation pour intrusion système.
\item Preuves: Logs, IP tracking, analyse forensique.
\item Décision: 2 ans prison, 5M FCFA amende.
\end{itemize}

\textbf{Affaire Ministère Public c. Y (2020)}:

\begin{itemize}
\item Cyber-escroquerie via mobile money.
\item Preuves: Analyse téléphonique, transactions.
\item Innovation: Utilisation de données USSD comme preuve.
\end{itemize}

\subsection{Défis Juridiques}
\begin{enumerate}
\item \textbf{Formation des magistrats}: Insuffisante en technique.
\item \textbf{Délais d'expertise}: 1-3 mois en moyenne.
\item \textbf{Coûts}: Expertise à la charge des parties.
\item \textbf{Standards}: Absence de normes nationales.
\end{enumerate}