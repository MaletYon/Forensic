\chapter{Les Grandes Affaires qui ont Façonné la Discipline}
\epigraph{"Chaque grande affaire forensic est un laboratoire où s'expérimente l'avenir de notre discipline."}{- Dr. Henry C. Lee}
\section{L'Affaire BTK Killer - Dennis Rader (2005)}
\textbf{Contexte}: Serial killer actif de 1974 à 1991, capturé grâce à des métadonnées

\textbf{Élément décisif}: Métadonnées d'un document Word sur disquette

\textbf{Leçon}: L'importance des métadonnées dans l'investigation

\textbf{Analyse technique détaillée}:

\begin{itemize}
\item Rader a envoyé une disquette à la police contenant un fichier "Test.A.rtf"
\item Les métadonnées révélaient: "Dennis" et "Christ Lutheran Church"
\item Utilisation de l'outil \textbf{ExifTool} aurait révélé les mêmes informations
\end{itemize}

\section{L'Affaire Stuxnet (2010)}
\textbf{Impact}: Première cyberarme reconnue publiquement

\textbf{Innovation}: Reverse engineering de malware industriel

\textbf{Techniques développées}:

\begin{itemize}
\item Analyse de code polymorphe
\item Identification de zero-days (4 utilisés)
\item Analyse comportementale en environnement sandboxé
\end{itemize}

\section{L'Affaire WannaCry (2017)}
\textbf{Envergure}: 300,000 ordinateurs dans 150 pays

\textbf{Héros}: Marcus Hutchins découvre le kill switch

\textbf{Innovation}: Analyse en temps réel d'une pandémie numérique

\textbf{Technique clé}: Analyse du Domain Generation Algorithm (DGA)