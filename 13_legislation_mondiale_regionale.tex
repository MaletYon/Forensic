\chapter{Législation Mondiale et Régionale}
\epigraph{"Cybercrime knows no borders, but our laws still do. International cooperation is not just desirable—it's imperative."}{- Susan W. Brenner}
\section{Droit Américain}
\subsection{Federal Rules of Evidence (FRE)}
\textbf{Rule 901 - Authentication}:

\begin{quote}
"To satisfy the requirement of authenticating or identifying an item of evidence, the proponent must produce evidence sufficient to support a finding that the item is what the proponent claims it is."
\end{quote}

\textbf{Application numérique}:

\begin{itemize}
\item Hash values acceptés (MD5 deprecated, SHA-256 minimum)
\item Chain of custody documentation requise
\item Expert testimony souvent nécessaire
\end{itemize}

\subsection{Stored Communications Act (SCA)}
\begin{itemize}
\item 18 U.S.C. §§ 2701-2712
\item Protection des communications stockées
\item Exceptions pour law enforcement avec warrant
\end{itemize}

\subsection{Computer Fraud and Abuse Act (CFAA)}
\begin{itemize}
\item 18 U.S.C. § 1030
\item Définit les cyber-crimes fédéraux
\item Base légale pour les investigations
\end{itemize}

\section{Droit Européen}
\subsection{Règlement eIDAS}
\textbf{Regulation (EU) No 910/2014}

\begin{itemize}
\item Signatures électroniques qualifiées
\item Horodatage qualifié
\item Services de confiance
\end{itemize}

\textbf{Niveaux de signature}:

\begin{enumerate}
\item \textbf{Simple}: Toute donnée électronique
\item \textbf{Avancée}: Identification unique
\item \textbf{Qualifiée}: Certificat qualifié + dispositif sécurisé
\end{enumerate}

\subsection{RGPD et Investigation}
\textbf{Règlement (UE) 2016/679}

\textbf{Tensions avec l'investigation}:

\begin{itemize}
\item Droit à l'effacement vs préservation de preuves
\item Minimisation des données vs collecte exhaustive
\item Notification de breach vs investigation secrète
\end{itemize}

\textbf{Article 23 - Limitations}:

Permet restrictions pour:

\begin{itemize}
\item Sécurité nationale
\item Prévention et détection d'infractions
\item Protection judiciaire
\end{itemize}

\subsection{Convention de Budapest}
\textbf{Convention sur la Cybercriminalité (2001)}

\begin{itemize}
\item 68 pays signataires
\item Harmonisation des législations
\item Coopération internationale
\end{itemize}

\textbf{Protocole additionnel 2021}:

\begin{itemize}
\item Divulgation directe par les ISPs
\item Accès transfrontalier d'urgence
\item Mutual Legal Assistance Treaty (MLAT) accéléré
\end{itemize}

\section{Droit Africain}
\subsection{Convention de Malabo (2014)}
\textbf{Convention de l'Union Africaine sur la Cybersécurité}

\textbf{Axes principaux}:

\begin{enumerate}
\item \textbf{Transactions électroniques}
\item \textbf{Protection des données}
\item \textbf{Cybercriminalité}
\item \textbf{Cybersécurité}
\end{enumerate}

\textbf{États parties}: 15 ratifications (sur 55 requis)

\subsection{Cadres Régionaux}
\textbf{CEDEAO}:

\begin{itemize}
\item Directive C/DIR/1/08/11 sur la cybercriminalité
\item Acte additionnel A/SA.2/01/10 sur les données personnelles
\end{itemize}

\textbf{SADC}:

\begin{itemize}
\item Model Law on Computer Crime and Cybercrime
\item Harmonisation en cours
\end{itemize}

\textbf{EAC}:

\begin{itemize}
\item Framework for Cyberlaws
\item Focus sur le commerce électronique
\end{itemize}